




\usepackage{kpfonts}
\usepackage{JWImports}
\usepackage{JWColors}



\mdfdefinestyle{envBoxBase}{
	linecolor = \baseColor!80,
	middlelinewidth = 2pt,
	roundcorner=5pt,
	backgroundcolor = \baseColor!5,
	innertopmargin=0.2cm,
	innerbottommargin=0.2cm,
	frametitlebelowskip=0cm,
	frametitleaboveskip=0cm,
	innerleftmargin=\topskip,
	innerrightmargin=\topskip,
	frametitlebackgroundcolor=\baseColor!80,
	frametitlefont=\color{white}\Large\bfseries,
	nobreak=true
}


\tikzset{
	warningsymbol/.style={
		rectangle,
		draw=\baseColor,
		fill=white,
		scale=1,
		overlay}
}

\mdfdefinestyle{algoBox}{
	linecolor = \baseColor!80,
	linewidth = 6pt,
	roundcorner=5pt,
	backgroundcolor = \baseColor!5,
	innertopmargin=0.2cm,
	innerbottommargin=0.2cm,
	frametitlebelowskip=0cm,
	frametitleaboveskip=0cm,
	innerleftmargin=\topskip,
	innerrightmargin=\topskip,
	topline=false,%
	rightline=false,%
	bottomline=false,%
	frametitlebackgroundcolor=\baseColor!80,
	frametitlefont=\color{white}\Large\bfseries,
	nobreak=true,
%	singleextra={\path let \p1=(P), \p2=(O) in ($(\x2,0)+0.5*(0,\y1)$) 
%		node[warningsymbol] {\danger};}
}


\newcommand{\JWAddNewTheoremStyle}[5]{
	\mdtheorem[style=#5]{#1}{#2}
	\newenvironment{#4}[1][]{
		\def\baseColor{#3}
		\begin{#1}[##1]
		}{\end{#1}}
}

\JWAddNewTheoremStyle{definitionEnv}{Définition}{indianred}{defn}{envBoxBase}
\JWAddNewTheoremStyle{stationEnv}{Station}{black}{station}{envBoxBase}
\JWAddNewTheoremStyle{activityEnv}{Activité}{chocolate}{activite}{envBoxBase}
\JWAddNewTheoremStyle{propEnv}{Propriété}{forestgreen}{prop}{envBoxBase}
\JWAddNewTheoremStyle{extypeEnv}{Algorithme}{blue}{algo}{algoBox}


\titlespacing{\chapter}{0cm}{0cm}{0cm}
\titleformat{\chapter}[display]
{\huge\bfseries}
{
	\chapterbox{#1}
}
{10pt}
{\bfseries}{\normalfont}

\titleformat{\section}[display]
{\huge\bfseries}
{
	\sectionbox{#1}
}
{10pt}
{\bfseries}{\normalfont}



\newcommand{\chapterbox}[1]{
	\def\chapIndent{2mm}
	\def\boxHeight{2cm}
	\begin{tikzpicture}[rounded corners,thick]
	\node (title) at (0,0) {\chaptername};
	\draw (title.west) --++(180:\chapIndent) --++(90:\boxHeight) --++(0:\textwidth) --++(-90:\boxHeight) -- (title.east);
	\node[anchor=west,fill=white,inner xsep=-2mm,xshift=\chapIndent,draw,yshift=\boxHeight/2] (chapternumber) at (title.east) {\fontsize{110}{132} \selectfont \thechapter};
	\node[anchor=west,](label) at (chapternumber.east) {\Huge#1};
	\end{tikzpicture}
}


\pgfdeclarelayer{bg}    % declare background layer
\pgfsetlayers{bg,main}  % set the order of the layers (main is the standard layer)
\newcommand{\sectionbox}[1]{
	\def\chapIndent{2mm}
	\def\boxHeight{1cm}
	\begin{tikzpicture}[rounded corners,thick]
	\node[anchor=west,fill=white,draw,inner xsep=0mm] (chapternumber) at (0,0) {\fontsize{40}{48} \selectfont \thesection};
	\begin{pgfonlayer}{bg}    % select the background layer
		\node[anchor=west,draw,xshift=-5mm](label) at (chapternumber.east) {\hspace*{5mm}\huge#1};
	\end{pgfonlayer}
	\end{tikzpicture}
}



















